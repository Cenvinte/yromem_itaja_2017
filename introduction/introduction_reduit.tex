\introduction
%Introduction

L’Institut de Formation et de Recherche en Informatique (IFRI) de l’Université d’Abomey-Calavi (UAC) a pour vocation de former des apprenants capables de devenir des acteurs pour des solutions informatiques aux différents problèmes de développement. Il offre des enseignements en continuelle adaptation à notre environnement social et comporte actuellement la licence en Génie Logiciel, la licence en Sécurité Informatique, la classe préparatoire de master et le Master de recherche. Le Master de recherche en informatique qui est proposé est réalisé en collaboration avec l'Université catholique de Louvain, premier partenaire de l'Institut dans son histoire.
\paragraph{}
Au terme de notre formation en Licence Professionnelle en Technologie de l'Information appliquée à la Gestion des Connaissances et Réseaux (LP-TIGCR) à l'IFRI qui s'est déroulé d'Octobre 2014 à Juillet 2015, il est demandé à tous les Étudiants de rédiger un mémoire de fin de formation au cours d'un stage pratique qui sera soutenu devant un panel de jury mis en place à cet effet.
\paragraph{} 
Conformément à cette exigence, nous avons effectué un stage pratique au Service Informatique de la Primature chargée du Développement Économique, de l'Evaluation des Politiques et de la Promotion de la Bonne Gouvernance (PDEEPPPBG) au cours duquel nous nous sommes intéressés au développement d'une application web pour la gestion des apprenants du Centre de Formation et de Technique de Gestion Documentaire dudit Ministère.
\paragraph{}
En effet, le PDEEPPPBG dispose d'un Centre de Formation en Informatique et en Techniques de Gestion Documentaire sous la tutelle par le Secrétariat Général du Ministère. Ce Centre de Formation a pour mission d'assurer la formation et le recyclage du personnel dudit Ministère, du personnel de l'administration public et des agents externes à l'administration publique, en secrétariat bureautique (Opérateur de saisie).
\paragraph{}
La gestion des apprenants depuis le dépôt des dossiers de candidature jusqu'à la proclamation des résultats de fin de formation en passant par le paiement des frais de formation se fait de façon manuelle. Aucune application automatisée n'a été mise en place pour gérer de façon efficace la formation des apprenants. c'est ce qui a suscité, au cours de notre stage, l'intérêt pour la question.
\paragraph{}
Le présent mémoire s'articule principalement autour de trois parties. Dans la première partie, il sera abordé le contexte général de l'étude. La deuxième partie est consacrée à l'analyse et la conception du système à l'aide du langage de modélisation dénommé \emph{Unified Modeling Language}(UML). Et enfin, la dernière partie présente le prototype de l'application Web réalisée pour résoudre les problèmes liés à la non automatisation de la gestion du Centre de formation.

%\begin{itemize}
%\item Premier \textbf{tiret gras}.
%\item Second \textit{tiret italique}.
%\item Troisième tiret.
%\end{itemize}

%\paragraph{Paragraph\\}

