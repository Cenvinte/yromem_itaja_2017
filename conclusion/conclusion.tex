\conclusion
Le présent travail de fin de formation intitulé : <<Conception et réalisation d'une plate-forme de gestion d'emplois du temps universitaire : cas de l'Université d'Abomey-Calavi (UAC)>> nous a permis de proposer une solution pour gérer les emplois du temps au sein du \gls{SGETPOIP} de l'\gls{UAC}.
Eu regard à notre sujet, nous nous sommes posé quelques questions ?
\begin{itemize}
\item[•] Comment proposer une solution opérationnelle pour aider le \gls{SGETPOIP}
\item[•] Quelles technologies choisir pour résoudre le problème
\item[•] Comment réduire le fossé (en terme de gestion d'emplois du temps) qui existe entre écoles et facultés
\end{itemize}
Sur ce, la mise en oeuvre d'une application Web pour la gestion d'emplois du temps a été le choix determinant pour résoudre tous les problèmes ci-haut.
Au cours de ce travail nous avons présenté les différentes étapes de la conception avec le puissant outil de modélisation \gls{UML} ainsi que la réalisation de l'application avec le Framework \gls{PHP} le plus populaire \textbf{Symfony}. S'il faut préciser quelque chose, ce projet nous a permis d'approfondir nos connaissances dans le domaine du développement Web.
\paragraph{}
Aussi, grâce à cette application la publication d'emplois du temps, leur gestion ainsi que le suivi des cours deviennent possibles et faciles.
\paragraph{}
Aux lecteurs de ce document, nous rappelons que cette application est loin d'avoir la prétention d'être parfaite. Elle n'est qu'une proposition de solutions aux problèmes de gestion d'emplois du temps universitaire. Nous restons ouvert aux critiques et suggestions.
\paragraph{}

