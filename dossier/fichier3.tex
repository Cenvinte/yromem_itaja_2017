\introduction

Au terme de l'analyse des données collectées qui a permis d'élaborer, entre autres, le diagramme de cas d'utilisation, le Diagramme des classes et le modèle logique des données(MLD), la dernière partie de cette étude est consacré à la présentation du prototype de l'application Web développée. 

\section{Présentation de l'outil de conception}

L'application a été développée avec les langages de programmation suivantes : PHP, MySQL, HTML, CSS, Java-script. La base de données a été conçue sous MySQL qui est un système de gestion de base données relationnelles (SGBDR). MySQL fait partie des logiciels de gestion de base de données les plus utilisés au monde, autant par le grand public (applications web principalement) que par des professionnels. Il fonctionne sous de nombreux systèmes d'exploitation et ses bases de données sont accessibles en utilisant les langages C, C++, C\#, VB, VB.NET, Java, Perl, PHP, etc. En outre de nombreuses entreprises dont Google, Yahoo! Youtube, Adobe, Airbus, Alston, Credit Agricole, Reuters, BBC News l'utilisent.			 

	\subsection{Langages utilisés}
		\begin{itemize}
			\item[•]PHP
	Hypertext Preprocessor, plus connu sous son sigle PHP (acronyme récursif), est un langage de programmation libre principalement utilisé pour produire des pages Web dynamiques via un serveur HTTP, mais pouvant également fonctionner comme n'importe quel langage interprété de façon locale. PHP est un langage impératif orienté objet comme C++.
PHP a permis de créer un grand nombre de sites web célèbres, comme Facebook, YouTube, Wikipedia, etc. Il est aujourd'hui considéré comme la base de la création des sites Internet dits dynamiques.

			\item[•]HTML
			L’Hypertext Markup Language, généralement abrégé HTML, est le format de données conçu pour représenter les pages web. C’est un langage de balisage permettant d'écrire de l’hypertexte, d’où son nom. HTML permet également de structurer sémantiquement et de mettre en forme le contenu des pages, d'inclure des ressources multimédias dont des images, des formulaires de saisie, et des programmes informatiques. Il permet de créer des documents interopérables avec des équipements très variés de manière conforme aux exigences de l’accessibilité du web. Il est souvent utilisé conjointement avec des langages de programmation (JavaScript) et des formats de présentation (feuilles de style en cascade). HTML est initialement dérivé du \emph{Standard Generalized Markup Language} (SGML).
			\item[•]CSS :
			Les feuilles de style en cascade, généralement appelées CSS de l'anglais \emph{Cascading Style Sheets}, forment un langage informatique qui décrit la présentation des documents HTML et XML. Les standards définissant CSS sont publiés par le \emph{World Wide Web Consortium} (W3C). Introduit au milieu des années 1990, CSS devient couramment utilisé dans la conception de sites web et bien pris en charge par les navigateurs web dans les années 2000.
			\item[•]JavaScript 
			est un langage de programmation de scripts principalement employé dans les pages web interactives mais aussi pour les serveurs. C'est un langage orienté objet à prototype, c'est-à-dire que les bases du langage et ses principales interfaces sont fournies par des objets qui ne sont pas des instances de classes, mais qui sont chacun équipés de constructeurs permettant de créer leurs propriétés, et notamment une propriété de prototypage qui permet d'en créer des objets héritiers personnalisés. En outre, les fonctions sont des objets de première classe.
Le langage a été créé en 1995 par Brendan Eich (Brendan Eich étant membre du conseil d'administration de la fondation Mozilla à cette époque) pour le compte de Netscape Communications Corporation. Le langage, actuellement à la version 1.8.2, est une implémentation de la 3e version de la norme ECMA-262 qui intègre également des éléments inspirés du langage Python. La version 1.8.5 du langage est prévue pour intégrer la 5e version du standard ECMA.

		\end{itemize}
\section{Architecture de l'application}
	\subsection{Captures d'écran}
		\begin{itemize}
			\item[•]Interface d'administration
		\end{itemize}
L'interface d'administration du site qui permet d'accéder à la gestion des modules, des formateurs, des apprenants, etc. comprend un nom d'utilisateur, un mot de passe et trois (3) boutons de commande (Ok, Effacer, Annuler). il se présente comme suit :
		\begin{figure}[H]
			\begin{center}
				\caption{Interface d'administration du site}
				\includegraphics[scale=1]{images/connexion.png}
			\end{center}
		\end{figure}						
		\begin{itemize}
			\item[•]Écran d'accueil
			\newline 
		L'écran d'accueil comporte six (6) menus que sont : Accueil, Formation, Candidature \& Inscription, Galerie Photos, Contactez-nous et Administration. L'écran d'accueil de l'application se présente ainsi qu'il suit :
		\begin{figure}[H]
		\begin{flushleft}
			\includegraphics[scale=1.0]{images/accueil.png}
			\caption{Écran d'accueil de l'application}
		\end{flushleft}
		\end{figure}
Le formulaire d'inscription des candidats se présente comme suit :
\begin{figure}[H]
		\begin{flushleft}
			\includegraphics[scale=0.75]{images/formulaire_candidat.png}
			\caption{Formulaire d'inscription du candidat}
		\end{flushleft}
		\end{figure}		

\section{Conditions minimales de déploiement de l'application}

Les conditions minimales de déploiement de l'application se présente ainsi qu'il suit :
 
\begin{itemize}
\item[-]un ordinateur de bureau ou portable qui répond aux caractéristiques suivantes : capacité disque : 80 Go, RAM : 2 Go, système d'exploitation : Windows 7 Pro. 64 bits. 
\item[-]installer un serveur EasyPHP 5.4.6
\item[-]installer Adobe Dreamweaver CS6
\item[-]ne peut fonctionner que dans un environnement Web. 
\end{itemize}

