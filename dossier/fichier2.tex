\introduction

En vue de doter le Centre de Formation en Informatique et en Techniques de Gestion Documentaire, d’une application web sécurisée, il convient d'analyser les données dans le cadre de notre étude. L'outil d'analyse choisi pour se faire est 'Unified Modeling Language' (UML). 
Pour recueillir les informations relatives à la gestion du Centre de formation, une interview a été réalisée avec le Chef dudit Centre qui a permis de cerner tout le contour du sujet et de décrire de façon claire le processus de gestion du Centre de Formation en Informatique. 
Le deuxième chapitre de ce rapport est consacré à l'étude de l'existant et à l'analyse et la conception du nouveau système. 

\section{Étude de l'existant : Description du processus de recrutement des apprenants}

Le processus de gestion des apprenants du Centre de formation en Informatique et en Techniques de Gestion Documentaire se déroule ainsi qu'il suit :
Le Chef dudit Centre élabore un prospectus à l'intention des futurs apprenants qu'il soumet à l'examen du Comité de Direction (CODIR) du Secrétariat Général du Ministère (SGM). Après examen et validation du prospectus par le CODIR du SGM, les corrections sont intégrées en vue d'obtenir la version finale qui est publiée et mis à la disposition des candidats en quête d'information sur les modalités pratiques d'inscription au Centre de formation. 
Les candidats intéressés par la formation déposent leurs dossiers de candidature et paient les frais d'étude du dossier et de test de sélection qui s'élèvent à la somme de 5.000 F CFA avant la date limite qui est généralement fixée à avant mi-novembre. Le dossier de candidature comprend les documents suivants :

\begin{itemize}

	\item[-]Une demande manuscrite adressée au Secrétariat Général du Ministère ;
	\item[-]Une copie d'acte de naissance ;
	\item[-]Une copie légalisée des diplômes ; 
	\item[-]Deux photos d'identité au format (3 cm x 3 cm) ; et enfin
	\item[-]Les frais d'études de dossier et de test de sélection.

\end{itemize}
Après le dépôt des dossiers, le Comité d'orientation du Centre étudie lesdits dossiers et en ressort les candidats qualifiés pour subir le test de sélection. Le test est organisé à la fin du mois de novembre de chaque année.
\paragraph{}
Les copies du test sont corrigées et après délibération, la liste des candidats retenus pour suivre la formation est publiée.
\paragraph{}

Après cette étape, en décembre la pré-rentrée est lancée au cours de laquelle, les apprenants paient les frais de formation qui s'élèvent à 130.000 F CFA et les frais des supports de cours d'un montant de 30.000 F CFA.
\paragraph{}

En janvier, les cours démarrent selon la programmation retenue par le Comité d'orientation du Centre de formation.
\paragraph{}

Les contrôles et évaluations se font de façon continue et au fur et à mesure que les modules sont dispensés aux apprenants par les formateurs qui sont pour la plupart des agents du Ministère en charge du Développement et du Ministère de l'Economie et des Finances.
\paragraph{}

Les copies des évaluations et des examens partiels sont corrigés par les formateurs et les notes sont transmis au Chef centre qui se chargent du calcul des moyennes selon les barèmes retenus. Le calcul de la moyenne finale tient compte de la note de stage. Après le calcul des moyennes qui comprend la note de stage, le Comité d'orientation délibère et sort la liste des apprenants admis. Ensuite une seconde session est organisée pour les candidats n'ayant pas obtenu une moyenne générale de 12 sur 20. Après correction des copies et délibération, la liste définitive des admis est affichée. Enfin, la cérémonie de remise des Diplômes est organisée au cours de laquelle les meilleurs apprenants sont récompensés. Certains d'entre eux se voient attribués des stages par l'Agence Nationale de Promotion de l'Emploi (ANPE). Il est à noter que le Diplôme délivré aux apprenants est reconnu par la Fonction Publique Béninoise. 

\section{Difficultés liées à la non-informatisation du Centre de Formation}
 
Comme nous l'avons mentionné dans la partie précédente, pratiquement tout le processus de gestion du Centre Informatique se fait de façon manuelle. La méthode actuelle de gestion du Centre Informatique du Ministère entraîne quelques difficultés qui peuvent se résumer ainsi qu'il suit. 
\paragraph{}
Le dépôt des dossiers des candidats est physique. Ceci pourrait entraîner des risques de perte de documents.  En outre, cette méthode favorise également, le dépôt des dossiers ou l'acceptation des dossiers en dehors du délai limite.
Les opérations de comptabilités se font également uniquement de façon manuelle. En cas de pertes de pièces comptables, étant donné qu'il n'y a aucune conservation des données par une application informatisée, il serait quasiment impossible de faire le bilan des opérations financières réalisées pour le compte du Centre de Formation.  
\paragraph{}
Concernant le calcul des moyennes, il se fait à travers le logiciel bureautique « Microsoft Excel ». Bien que cette méthode peut être qualifié de semi-informatisé, elle est laborieuse et harassante en raison du fait que l'on doit reprendre les noms et prénoms des apprenants pour chaque feuille de calcul Excel.
\paragraph{} 
A l'heure où le monde est devenu un village planétaire en raison du développement et l'expansion à un rythme exponentiel des Nouvelles Technologies de l'information et de la Communication, les candidats sont obligés de se déplacer au Ministère pour voir les résultats des tests, ce qui constitue une perte d'énergie, de temps et d'argent pour eux. Or, avec une application web, ils auraient pu rien qu'avec leur ordinateur ou leur téléphone portable consulter les résultats aussi bien de la liste des candidats retenus que des résultats du test de sélection ainsi que les résultats de fin de formation. 

\section{Analyse et conception du nouveau système}

Conformément au prospectus (cf. annexe) publié par le Centre de formation, il est exigé des apprenants les documents suivants :
Une demande manuscrite ; une copie de l'acte de naissance ; une copie légalisée des diplômes ; deux photos d'identité ; et le paiement des frais d'étude de dossier et de test.
\paragraph{}
A partir de ces informations, nous pouvons déduire que les attributs des candidats sont : les nom et prénoms, la date de naissance, le lieu de naissance, le niveau d'étude ou diplômes obtenus, la photo d'identité.
\\
Pour plus d'efficacité dans la conception, il s'avère important d'y ajouter un numéro d'inscription, le sexe, l'adresse électronique et l'adresse postale.
En outre, les modules enseignés sont les suivants :
(i) Windows; (ii) Dactyl; (iii) Word; (iv) Excel; (v) Access; (vi) Outlook et Internet; (vii) Powerpoint; (viii) Publisher; (ix) Essentiel de secrétariat; (x) Correspondance administrative; et (xi) Classement. 
 
\paragraph{}
Ce qui devrait nécessairement orienter à la création d’un objet pour la gestion des modules.
Par ailleurs, le formateur constitue un acteur principal dans la dispense des cours au sein du Centre de formation. Car, ils dispensent les modules, composent les devoirs et examens et attribuent les notes. Il convient donc de créer un objet 'Formateur' qui aura les attributs suivants : matricule, nom et prénoms, date de naissance, lieu, sexe, téléphone, adresse postale, email.
Dans cette analyse avec l'outil UML, trois (3) diagrammes seront réalisées. Il s'agit de : le diagramme des cas d'utilisation, le diagramme de séquence et le diagramme de classe. Avant de présenter, ces différents diagrammes issus de notre analyse, une brève introduction sera faite de l’outil UML.
\section{L'Outil d'analyse 'Unified Modeling Language' (UML)}

Le langage de modélisation unifié, de l'anglais Unified Modeling Language (UML), est un langage de modélisation graphique à base de pictogrammes conçu pour fournir une méthode normalisée pour visualiser la conception d'un système. Il est couramment utilisé en développement logiciel et en conception orientée objet.
UML est le résultat de la fusion de précédents langages de modélisation objet : Booch, OMT, OOSE. Principalement issu des travaux de Grady Booch, James Rumbaugh et Ivar Jacobson, UML est à présent un standard adopté par l'Object Management Group (OMG).
UML est utilisé pour spécifier, visualiser, modifier et construire les documents nécessaires au bon développement d'un logiciel orienté objet. UML offre un standard de modélisation, pour représenter l'architecture logicielle. Les différents éléments représentables sont :
\begin{itemize}
	\item[-]Activité d'un objet/logiciel
    \item[-]Acteurs
    \item[-]Processus
    \item[-]Schéma de base de données
    \item[-]Composants logiciels
    \item[-]Réutilisation de composants
\end{itemize}

UML 2.3 propose 14 types de diagrammes (9 en UML 1.3). Dans le cadre de la présente étude, trois (3) diagrammes seront présentés.
\subsection{Diagramme de cas d'utilisation}
Les diagrammes de cas d'utilisation sont des diagrammes UML utilisés pour donner une vision globale du comportement fonctionnel d'un système logiciel. Ils sont utiles pour des présentations auprès de la direction ou des acteurs d'un projet, mais pour le développement, les cas d'utilisation sont plus appropriés. Un cas d'utilisation représente une unité discrète d'interaction entre un utilisateur (humain ou machine) et un système. Il est une unité significative de travail. Dans un diagramme de cas d'utilisation, les utilisateurs sont appelés acteurs (actors), ils interagissent avec les cas d'utilisation (use cases).
\newline
L'analyse du processus de gestion du Centre de formation, a permis de recenser les différents cas d'utilisation suivants :
\begin{itemize}
	\item[i]créer compte utilisateur
	\item[ii]s'authentifier
	\item[iii]s'inscrire en ligne
	\item[iv]publier le prospectus
	\item[v]consulter prospectus
	\item[vi]payer frais formation
	\item[vii] délivrer reçu paiement
	\item[viii]publier liste Candidature retenue
	\item[ix]élaborer programme formation
	\item[x]éditer bulletin de notes
	\item[xi]proclamer résultat
\end{itemize} 

Trois acteurs interviennent dans ce module UML. Il s'agit de l'apprenant, du candidat et de l'administration du centre de formation.
Ainsi, se présente le diagramme des cas d'utilisation globale du système de gestion des apprenants du Centre de formation, réalisé avec Microsoft Visio Professionnel 2013 :
\begin{figure}[H]
\begin{center}
	\includegraphics[scale=0.75]{images/userCaseDiagram.png}
	\caption{Diagramme de cas d'utilisation}
\end{center}
\end{figure}

\subsubsection{Description nominale de quelques cas d'utilisation}
	\begin{itemize}
	\item \textbf{Cas d'utilisation : Créer compte utilisateur}
		\item[-]Nom : Créer un compte d'utilisateur pour un Candidat 
		\item[-]Acteurs Concernés : candidat
		\item[-]Description succincte : le candidat crée son compte 			utilisateur en vue d'une éventuelle inscription en ligne.
		\item[-]Date : 21/07/2015 (première rédaction)
		\item[-]Auteur : BISSIRIOU Mohamed Loukouman Adio.
		\item[-]Pré-conditions : l'utilisateur doit lancer le 					navigateur, taper l'adresse du site Web du Centre de 					formation et choisir l'option « créer compte utilisateur ».
		\item[-]Démarrage : L'utilisateur doit faire un clic sur la 			commande 'Créer compte utilisateur'
	\end{itemize}
	
	\begin{bf}
		Description
		\newline
		Le scénario nominal
	\end{bf}
	\begin{itemize}
	\item[1-]le système affiche la page d'accueil du site web.
	\item[2-]L'utilisateur sélectionne la commande créer compte utilisateur.
	\item[3-]Le système ouvre une page et offre la possibilité à 		l'utilisateur de saisir un nom et un mot de passe.
	\item[4-]L'utilisateur saisit un nom, un mot de passe puis valide les informations saisies. 
	\item[5-]Le système vérifie si le nom saisi n'existe pas déjà dans la base de données.
	\item[6-]Le système crée un compte utilisateur.
	\item[7-]L'utilisateur peut ensuite quitter la page ou continue en s'authentifiant en vue d'une inscription en ligne.
	\end{itemize}
	\begin{bf} 
	Les scénarios alternatifs 
	L'utilisateur décide de quitter la page de création de compte
	\end{bf}
	\begin{itemize}
			\item[a-]le nom utilisateur saisi par l'utilisateur n'est pas valide. (Existe déjà)
	\item[b-]le système ne crée pas de compte utilisateur.
	\item[c-]l'utilisateur peut quitter la page ou reprendre la procédure d'inscription.
	
	\end{itemize}
	
	
	\begin{itemize}
	\item \textbf{cas d'utilisation : s'authentifier}
	\item[-]Nom : s'authentifier
	\item[-]Acteurs Concernés : candidat
	\item[-]Description succincte : le candidat s'authentifie à travers son nom d'utilisateur et son mot de passe avant de s'inscrire en ligne.
	\item[-]Date : 23/07/2015 (première rédaction)
	\item[-]Auteur : BISSIRIOU Mohamed Loukouman Adio
	\item[-]Pré-conditions : le candidat doit créer un compte avant de 	s'authentifier. 
	\item[-]Démarrage : L'utilisateur doit faire un clic sur la commande 'se connecter'
	\end{itemize}
	\begin{bf}
	Description 
	\newline
	Le scénario nominal
	\end{bf}
	\begin{itemize}
	\item[1-]le système affiche la page d'accueil du site web.
	\item[2-]L'utilisateur sélectionne la commande se connecter sur la page d’accueil.
	\item[3-]Le système ouvre une page et offre la possibilité à l’utilisateur de saisir un nom et un mot de passe.
	\item[4-]L'utilisateur saisit son nom et son mot de passe puis valide les informations saisies. 
	\item[5-]Le système vérifie si le nom saisi et le mot de passe sont valides
	\item[6-]Le système ouvre le formulaire d'inscription du candidat.
	\item[7-]L'utilisateur peut s'inscrire en ligne ou quitter la page.
	\end{itemize}
	\begin{bf}
		Description
		Les scénarios alternatifs 
	\end{bf}
	L'utilisateur décide de quitter la page de création de compte
	\begin{itemize}
		\item[]5.a) le nom utilisateur saisi par l'utilisateur n'est pas valide. (Existe déjà)
	\item[]6.a) le système ne crée pas de compte utilisateur.
	\item[]7.a) l'utilisateur peut quitter la page ou reprendre la procédure d'inscription.
	\end{itemize}
	
	\begin{bf}
	 .cas d'utilisation proclamer résultats de la formation
	\end{bf}

	\begin{itemize}
	\item[-]Nom : proclamer résultats de la formation 
	\item[-]Acteurs Concernés : système, utilisateur.
	\item[-]Description succincte : l'utilisateur à travers une requête, 	sélectionne les apprenants ayant obtenu la moyenne requise (12) et proclame les résultats des admis.
	\item[-]Date : 29/07/2015 (première rédaction)
	\item[-]Auteur : BISSIRIOU Mohamed Loukouman Adio
	\item[-]Pré-conditions : l'utilisateur doit s'authentifier avant toute consultation.
	\item[-]Démarrage : L'utilisateur demande la page de consultation des résultats.
	\end{itemize}
	\begin{bf}
	Description 
	Le scénario nominal
	\end{bf}
	\begin{itemize}
		\item[1-]le système affiche la page de consultation.
		\item[2-]L'utilisateur entre les données relatives à la requête et valide.
		\item[3-]Le système affiche les résultats de la requête.
		\item[4-]L'utilisateur exploite ses résultats et les publie sur le site Web.
		\item[5-]Le système affiche dans une page web les résultats de fin d'année.
	
	\end{itemize}
		
		\begin{bf}
		Les scénarios alternatifs
		\end{bf}
		
		L'utilisateur saisi des données de recherche invalides
		\begin{itemize}
			\item 2.a) l'utilisateur décide de quitter les données de la consultation. 
			\item 4.a) l'utilisateur décide de quitter la page de la consultation.
		
		\end{itemize}
		
		
\subsection{Diagramme de séquence}

Les diagrammes de séquences sont la représentation graphique des interactions entre les acteurs et le système selon un ordre chronologique dans la formulation Unified Modeling Language.

L’analyse des données a permis de proposer le diagramme de séquence suivant réalisé avec le logiciel Visual Paradigm Enterprise Version 11.1 :

\begin{figure}[H]
\begin{center}
	\includegraphics[scale=1.0]{images/diagram_sequence.png}
	\caption{Diagramme de séquence}
\end{center}
\end{figure}

\subsection{Diagramme de classe}

Le diagramme de classes est un schéma utilisé en génie logiciel pour présenter les classes et les interfaces des systèmes ainsi que les différentes relations entre celles-ci. Ce diagramme fait partie de la partie statique d'UML car il fait abstraction des aspects temporels et dynamiques.
Une classe décrit les responsabilités, le comportement et le type d'un ensemble d'objets. Les éléments de cet ensemble sont les instances de la classe.
Dans le cadre de la présente étude, l'analyse des données a permis d'obtenir les sept (7) classes suivantes :
\begin{itemize}
	\item[1.]La classe Module ;
	\item[2.]La classe Apprenant ;
	\item[3.]La classe Candidat ;
	\item[4.]La classe Formateur ; 
	\item[5.]La classe Évaluer ; 
	\item[6.]La classe Promotion ; et 
	\item[7.]La classe Enseigner.
\end{itemize}
 
\subsubsection{Description des associations entre les classes}
Les sept classes du diagramme des classes sont reliées à travers 4 associations.
Il existe une association de type un vers plusieurs entre la classe Apprenant et la classe Promotion.  Un apprenant est inscrit dans une et une seule promotion, pendant que dans une promotion, il y'a plusieurs apprenants. 
\paragraph{}
La classe Apprenant hérite de la classe Candidat. Alors que entre la classe Formateur, la classe Module et la classe Promotion, il existe une relation ternaire de plusieurs à plusieurs porteuse de propriétés (Masse horaire, Période, Horaire) qui donne naissance à la Classe Enseignement.
\paragraph{} 
La quatrième relation est une relation de plusieurs à plusieurs entre Module et Apprenant qui donne naissance à la classe Évaluation dont les attributs sont : Note1, Note2, Note3, Note4.
\paragraph{}
La figure suivante présente le diagramme des classes réalisé avec Visual Paradigm Enterprise Edition 11.1.
\begin{figure}[H]
\begin{center}
	\includegraphics[scale=1.0]{images/class_diagram.png}
	\caption{Diagramme de classes}
\end{center}
\end{figure}

\section{Modèle logique des Données(MLD)}
A partir du diagramme des classes réalisé, en application des règles de passage au MLD, nous obtenons les tables suivantes :
\newline
	\begin{itemize}
		\item{-}Candidat(numInscription, nom, prenoms, dateNaissance,sexe,email, contacts, adresse, photo)
		\item{-}Apprenant (codeApprenant,codePromo)
		\item{-}Promotion (codePromo)
		\item{-}Formateur (matricule, nom, prenoms, profile, email, contact)
		\item{-}Module(codeModule, libelle)
		\item{-}Evaluation(\#codeModule,\#codeApprenant, note1, note2, note3)
		\item{-}Enseignement(\#codePromo, \#matricule, \#codeModule, horaire, periode, masseHoraire)
	
	\end{itemize}
	


		

	



