\introduction 

La Primature est chargée du Développement Économique, de l'évaluation des Politiques Publiques et de la Promotion de la Bonne Gouvernance (PDEEPPPBG). C'est un Ministère transversal qui a pour mission d'impulser et de piloter le développement économique et social du Bénin. Il dispose d'un Centre de Formation en Informatique et en Techniques de Gestion Documentaire sous la tutelle du Secrétariat Général du Ministère. Ce centre a pour vocation de renforcer les capacités du personnel dudit Ministère en formant les agents à l'utilisation de logiciels spécifiques nécessaires à l'exécution des tâches quotidiennes  d'une part et d'autre part de former des usagers externes au Ministère pendant une durée de 12 mois (9 mois de formation et 3 mois de stage pratique). Les modules enseignés au cours de cette formation sont : Windows, Microsoft Office, Correspondance Administrative, Dactylographie et Techniques de base du secrétariat. L'année académique 2015 a accueilli la 42ème promotion du Centre de Formation. Et depuis sa création, la gestion du Centre de formation se fait de façon manuelle. En d'autres termes, aucune application informatisée n'a été conçue à ce jour pour automatiser la gestion du Centre de Formation. Seul le calcul des moyennes des apprenants se fait avec le tableur Microsoft Excel.
Dans le présent chapitre, nous essayerons de décrire le processus de gestion des apprenants du Centre de Formation en Informatique et en Techniques de Gestion Documentaire. Ensuite, nous allons donner les difficultés et conséquences liées à cette gestion manuelle.
\section{Présentation de la direction d'accueil}

Au terme de notre formation à l'Institut de Formation et de Recherche en Informatique (IFRI) qui s'est déroulée d'Octobre 2014 à Juillet 2015, nous avons effectué un stage pratique à la Direction de l'Informatique et du Pré-archivage (DIP) de la Primature chargée du Développement Économique, de l'Evaluation des Politiques Publiques et de la Promotion de la Bonne Gouvernance (PDEEPPPBG), plus précisément au Service Informatique. 

%En effet, le Service Informatique du PDEEPPPBG est rattaché au Secrétariat Général du Ministère qui comprend :
La Primature a pour mission d'impulser le développement économique et social, d'assurer le suivi de la mise en œuvre des politiques, actions et décisions du Gouvernement en matière de développement économique, d'évaluation des politiques publiques et de promouvoir la Bonne Gouvernance.

La Primature comprend :
\begin{itemize}
	\item[-]le Premier Ministre ;
	\item[-]les services et personnes directement rattachés au Premier Ministre ;
	\item[-]le Cabinet du Premier Ministre ; 
	\item[-]le Secrétariat Général de la Primature ; 
	\item[-]l'Inspection Générale de la Primature ;
	\item[-]les Directions Centrales ;
	\item[-]les Directions Techniques et les Directions Départementales ;
	\item[-]les Organismes sous-tutelle.
\end{itemize}

Au nombre des directions centrales de la Primature figure la Direction de l'Informatique et du Pré-archivage (DIP) qui assure, en relation avec toutes les structures de la Primature la conception, la mise en œuvre, la coordination et le suivi-évaluation d'actions intégrées visant à :
\begin{itemize}
	\item[-]garantir la sécurisation formelle, l'authentification et la sauvegarde des documents administratifs et autres productions intellectuelles ;
	\item[-]assurer l'accessibilité et la qualité des nouvelles technologies de l'information et de la communication au sein de la Primature ;
	\item[-]faciliter les relations entre les directions techniques et les usagers/clients pour un service public efficace et efficient.
\end{itemize}
La Direction de l'Informatique et du Préarchivage (DIP) comprend :
\begin{itemize}
	\item[-]le Secrétariat ;
	\item[-]le Service des Relations avec les Usagers ;
	\item[-]le Service de Pré-archivage et de Gestion des Savoirs ;
	\item[-]le Service Informatique.
\end{itemize}

Le Service Informatique qui nous a accueilli pour le stage comprend trois (3) divisions à savoir : (i)la Division de la Conception des Systèmes d'information; (ii) la Division de la Maintenance; et (iii) la Division de l'Administration du Réseau.

L'organigramme de la Primature est présenté dans le schéma ci-après :
\begin{center}
	\begin{figure}
		\includegraphics[scale=1.0]{images/organigram_mdaep.png}
		\caption{Organigramme de la Primature chargée du Développement Économique, de l'Evaluation des Politiques Publiques et de la Promotion de la Bonne Gouvernance}
	\end{figure}
\end{center}
%Organigramme de l'ex-Ministère de Développement, de l'Analyse Économique et de la Prospective (MDAEP)
  
\section{Cadre général de l'étude}
	\subsection{Présentation générale du projet d'étude}
Chaque année, au cours du mois de Novembre, la Primature chargé du Développement économique, de l'évaluation des politiques publiques et de la promotion de la bonne gouvernance recrute pour le compte du Centre de Formation en Informatique et en Technique de Gestion Documentaire (CFI-TGD) à travers un test écrit de sélection, des apprenants ayant au moins le niveau 3ème pour suivre une formation dans les modules suivants :
\begin{itemize}
	\item[-]Suite Bureautique Microsoft (word, excel, access, powerpoint, publisher, outlook) ; 
	\item[-]Dactylographie ;
	\item[-]Secrétariat ;
	\item[-]Correspondance administrative ; et
	\item[-]Classement des documents. 
\end{itemize}
\paragraph{}
Toutes les activités du Centre de formation depuis sa création en 1989 se font de façon manuelle. A l'exception du calcul des moyennes qui se fait à l'aide du tableau Microsoft Excel. 
Le présent projet d'étude vise à doter le Centre de Formation en Informatique et en Technique de Gestion Documentaire de la Primature d'une plateforme Web de gestion des admissions en ligne des apprenants externes et des résultats de fin de formation des apprenants. 
\subsection{Problématique}
Dans le cadre de ses activités de formation des usagers externes au Ministère, le Centre de Formation en Informatique et en Techniques de Gestion Documentaire, recrute chaque année environs 70 apprenants qu'il forme pendant 12 mois. Dans l'exécution de ses activités, le Centre rencontre quelques difficultés liées au caractère rudimentaire qui sont utilisé dans la réalisation des tâches. En effet, le recrutement, la proclamation des résultats de test, la programmation des résultats de fin de formation sont exécutées de façon manuelle. Cette méthode a pour conséquence, la lenteur dans l'exécution des tâches. En outre, le calcul des moyennes bien que se faisant avec Excel est laborieux puisque toutes données doivent à chaque fois se reproduire pour la saisie des notes et le calcul des moyennes proprement dit. 
\section{Objectifs}
\subsection{Objectif général}
Le présent projet de fin d'étude vise à mettre à la disposition du Centre de Formation en Informatique et en Technique de Gestion Documentaire une application Web sécurisée de gestion en ligne de ses apprenants externes depuis le recrutement jusqu'à la proclamation des résultats de fin de formation.
\subsection{Objectifs spécifiques}
De façon spécifique, ce projet vise à doter le Centre de formation d'une solution répondant aux besoins fonctionnels suivants :
\begin{itemize}
	\item[-]Permettre l'inscription en ligne des apprenants externes 	;
	\item[-]Gérer de façon automatisée les résultats des tests de sélection des apprenants ;
	\item[-]Gérer les informations utiles sur les formateurs du Centre ;
	\item[-]Planifier le calendrier de la formation ;
	\item[-]Gérer les modules de formation et leurs masses horaires ;
	\item[-]Éditer les bulletins de notes des apprenants.
\end{itemize}




