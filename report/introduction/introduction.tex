\introduction

\paragraph{}
  Des ventes de mains en mains vers des ventes virtuelles, l'environnement fortement concurrentiel des entreprises n'autorise aucune erreur de gestion. La plupart des entreprises des pays en voie de développement tant du secteur public que privé font malheureusement face à des difficultés énormes de gestion tant celui des biens, des services, des employés que des boutiques.

\paragraph{}
  Les solutions informatisées de gestion de stocks sont depuis des années, largement conseillées pour les sociétés qui font du commerce leur activité principale. Ces solutions représentent un dispositif global fournissant aux commerçants et propriétaires de boutiques une plateforme où se retrouvent l'inventaire des entrées de marchandises, des ventes etc..

\paragraph{}
  Le suivi des flux permet aux commerçants et propriétaires de boutiques de connaître les produits les plus vendus, les boutiques qui reçoivent plus de clientèle. La courbe évolutive de leur chiffre d'affaires doit être disponible et quotidiennement mise à jour sans la moindre erreur, ce qui leur guiderait dans leurs investissements et le choix de leurs produits principaux.

\paragraph{}
  Notre projet, réalisé dans le cadre du mémoire de licence professionelle intitulé : \textbf{<<ITAJA : Système intégré de gestion d’un commerce>>} se veut une solution complète et efficace des problèmes cités ci-haut. Nous allons dans premier temps analyser le problème et ensuite proposer une solution pour organiser de façon optimale la gestion des stocks d'un commerce.