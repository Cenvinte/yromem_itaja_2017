\resume
\begin{abstract}
%Résumé en français
\paragraph{}
%\textbf{Mots clés}: \emph{Mot clé 1}, \emph{Mot clé ...}, \emph{Mot clé N}.

Ce mémoire de fin de formation se propose de trouver une solution aux problèmes liés à la gestion manuelle du Centre de formation en Informatique et en Techniques de Gestion Documentaires de la Primature chargée du développement économique, de l'évaluation des politiques publiques et de la Promotion de la Bonne Gouvernance (PDEEPPPBG) en vue de la rendre plus performante et par conséquent améliorer la qualité des services offerts aux apprenants dudit Centre. En effet, depuis sa création en 1989, la gestion des activités dudit Centre n'a connu aucun début d'automatisation, ce qui entraîne un certain nombre de difficultés dans sa gestion à savoir : le caractère laborieux et fastidieux des tâches à réaliser, le manque de célérité dans les tâches, la perte de temps, d'énergie des candidats potentiels obligés de se déplacer vers le Centre pour avoir des informations relatifs aux conditions de recrutement, aux résultats des test, etc.

\paragraph{}
Pour pallier les difficultés sus-mentionnées la présente étude propose, au Centre de Formation en Informatique, une application web qui répond aux fonctionnalités suivantes : gérer les apprenants du centre, gérer les formateurs du Centre, gérer les modules de la formation, gérer les notes des apprenants, calculer les moyennes des étudiants, imprimer les bulletins des apprenants, consulter les notes des apprenants, publier le calendrier de la formation, etc.
En outre, cette application Web permettrait également aux candidats potentiels de s'inscrire en ligne après avoir créé un compte et s'être authentifier.

\paragraph{}
En raison, du temps relativement cours consacré à la rédaction et la conception de l'application, toutes les fonctionnalités n'ont pas été prises en compte dans la présente étude. Les réflexions se poursuivront après le dépôt des rapports de mémoire pour compléter les autres fonctionnalités contenues dans le cahier de charges de la présente étude avant la soutenance.

\end{abstract}
\newpage
\selectlanguage{english}
\begin{abstract}
%Résumé en anglais
\paragraph{}
%\textbf{Key words}: \emph{Key word 1}, \emph{Key word ...}, \emph{Key word N}.
This report of the end of training course try to find solution to the problems due to the manual management of the Center of Computer Science and technics of document's management of the Ministry in charge of economic development, evaluation of public policies and promotion of good governance in order to make it more effective and consequently to improve the quality of services offered to the students of the Center. In fact, since his creation in 1989, the management of the Center is not computerized. This situation cause some dificulties in the mamagement such as lack of celerity in the performing of the tasks, wasting of time and energy, potential candidates are obliged to visit the Center to get informations about the admission criteria, the results of the tests, etc.

\paragraph{}
The actual study suggests some proposals of solutions to the Center of computer training through a web application which fulfil the following functionalities : to manage the students, to manage the teachers, to manage the modules, to manage the students's marks, to carry out the students's average, to search the students's marks, to publish the training course planning, etc.
Moreover, this web application will offer to the potential students opportunity to apply online after creating an personal account and authentification.

\paragraph{}
Due to the short period given for the writing of the report and the conception of the application, all the functionalities are not been implemented and mentioned in the present study. The reflection will be pursued after the deposit of the report so that to complete the others functionalities contained in the conditions specified before the defence. 
\end{abstract}
