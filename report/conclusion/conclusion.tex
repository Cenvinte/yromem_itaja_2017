\conclusion
Le présent travail de fin de formation intitulé «ITAJA: Système intégré de gestion d’un commerce» nous a permis de proposer aux commerçants et propriétaires de magasins une solution pour gérer leurs stocks(inventaires), les opérations effectuées (ventes, dépenses), et surtout l'évolution des statistiques financières (chiffre d'affaires, charges), et ce à moindre coût. Eu regard à notre sujet, nous nous sommes posé quelques questions à savoir :

\begin{enumerate}
  \item[•] Comment proposer une solution opérationnelle pour les propriétaires de magasins
  \item[•] Quelles technologies choisir pour résoudre le problème.
\end{enumerate}

Sur ce, la mise en oeuvre d’une application Web pour la gestion des magasins a été le choix determinant pour résoudre tous les problèmes ci-haut. Au cours de ce travail nous avons présenté les différentes étapes de la conception ainsi que la réalisation de l’application. S’il faut préciser quelque chose, ce projet nous a permis d’approfondir nos connaissances dans le domaine du développement Web.

\paragraph{}
  Aux lecteurs de ce document, nous rappelons que cette application est loin d’avoir la prétention d’être parfaite. Elle n’est qu’une proposition de solutions aux problèmes de gestion de stocks et de magasins pour les commerçants et propriétaires de magasins. Nous restons ouvert aux critiques et suggestions.